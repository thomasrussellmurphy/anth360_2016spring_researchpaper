\RequirePackage{snapshot}
\documentclass[letterpaper, 12pt]{article}
\input{default}

\usepackage[pdftex,
            pdfauthor={Thomas Russell Murphy},
            pdftitle={Medical Training for Abortion Procedures and Physician Provision of Abortions in the United States},
            pdfsubject={},
            pdfkeywords={abortion, training, medical training, OB/GYN, obstetrician, gynecology, reproduction, fertility, birth control},
            pdfproducer={\pdftexbanner},
            pdfcreator={{\pdftexbanner} with Biber and a Makefile}]{hyperref}

% Setup Chicago 16th edition, footnotes-mode citations
\usepackage[notes,short,backend=biber]{biblatex-chicago}
\bibliography{exported_items_active_updating}

% Use English babel and fancy quotes
\usepackage[american]{babel}
\usepackage{csquotes}

\title{Medical Training for Abortion Procedures and Physician Provision of Abortions in the United States}
\author{Thomas R. Murphy}
\date{\today\\ANTH 360}

\usepackage{titlesec}
\titleformat*{\section}{\large\bfseries}

\begin{document}
\maketitle

% Original general thesis: "Abortion training is becoming more available and being completed by more physicians, but this trend needs to continue so that future practicing physicians can provide better access to care, particularly with regard to primary care."
% Revised thesis points:
%   - training is not the only barrier
%   - policy influence on care provided due to individual and organizational beliefs

\section*{Introduction}

% Background on abortion in the United States
% Contextualizing these providers of abortion
The landscape of abortion in the United States is fairly evident in its imposition of restrictions of access to procedures for patients, but barriers also exist for providers.
Abortion is a common operation in the United States, with an estimated 1.1 million abortions performed in 2011. \autocite[pg. 7]{jones_abortion_2014}
However, geographic access to abortion is not uniform: in 2011, 89\% of US counties had no abortion or nonspecialized clinic with 38\% of women aged 15-–44 lived in those counties, essentially unchanged from the geographic availability in 2008. \autocite[pg. 7]{jones_abortion_2014}
More specifically, the division of abortion providers is distinct between urban centers and rural areas.
While the presence of abortion providers is more concentrated in cities, in 2008, 69\% of counties in metropolitan areas lacked a provider while 97\% of nonmetropolitan counties lacked a provider. \autocite[pg. 46]{jones_abortion_2011}

The geographic disparity in access, however, does not correspond to a disparity in demand for abortion services.
A 2008--9 survey found that 97\% of obstetrician-gynecologists (OB/GYNs) in the United States under the age of 65 encountered patients seeking abortions while only 14.4\% of those practicing OB/GYNs provided abortions themselves. \autocite[pg. 611]{stulberg_abortion_2011}
In the United Stater, some 20,090 OB/GYNs were employed in May 2015, primarily in physicians' offices and outpatient care centers. \autocite{occupational_employment_and_wages_may_2015}
The primary places of employment are not primary centers for the provision of abortion services, though.
While 19\% of abortion providers in 2008 were physicians' offices, they only provided 1\% of all abortions at the greatest expense and at a low rate per location. \autocite[pg. 46--7]{jones_abortion_2011}
In comparison to the number of abortions in the United States, 32.8\% of births were Cesarean deliveries, about 1.3 million births in 2011. \autocite[pg. 5]{martin_births_2013}
This more complex prodedure is distinctly more accepted than abortion, but has not generated the same magnitude of controversy despite its higher frequency of use.

As I will discuss, abortion training is becoming more available for and being completed by more physicians, but improving training does not remove all systemic barriers preventing physicians from providing abortion. I will discuss the circumstances of abortion training during OB/GYN residency, barriers to providing abortion services once physicians enter practice, and difficulties with integrating abortion into the procedures offered by routine healthcare providers. Throughout, there are policies and communication influenced by the stigma of or opposition to abortion that negatively impact physicians ability to provide care and patients to access the care they desire or need.

% Discussion of OB/GYN *training* in residency
\section*{Abortion Training Access in OB/GYN Residency}

While abortion has had a long history \autocite[For a thesis about training for both procedural and elective abortions in the United States from 1920 to 2007, see:][]{ayres_providing_2009} in the United States, I will discuss more contemporary changes in training and practice.
% Changing environment of abortion healthcare
In 1973, the decision of \textit{Roe~v~Wade} enabled abortions to shift out of hospitals and into non-hospital clinics, thus generally removing the common procedure and its associated curricula from the US medical educational environment. \autocite[pg. 404]{aksel_unintended_2013}
Although the shift to abortion clinics expanded access to elective abortions, the overall impact on female health had negative impacts in the long term.
Even though the media asked if abortion should have been a solved issued 10 years after \textit{Roe}\autocite{mark_obenhaus_abortion_1983}, it is evident that acceptance of abortion as an acceptable, safe, and routine medical operation.
Another ten years later, in the early 1990s, there was a marked lack of training in family planning, including abortion, but in 1994 Jody Steinauer started the meetings with fellow medical students that became the organization Medical Students for Choice. \autocite[pg. 404--5]{aksel_unintended_2013}

Part of the transformation in abortion training was the change by the Accreditation Council for Graduate Medical Education (ACGME) to mandate availability of or access to abortion training in OB/GYN residency. \autocite[IV.A.6.d, pg. 17]{accreditation_council_for_graduate_medical_education_acgme_2014}
First mandated in 1996 \autocite[pg. 146]{freedman_obstacles_2010}, the requirements have remained in place.
While this mandate regarding access to curriculum may improve training, it is also advantageous to have comprehensive, structured family planning and abortion training rather than merely procedural, routine abortion training. \autocite[pg. 297]{macisaac_routine_2012}

% Enablers, barriers, consequences
Within medical education programs with OB/GYN residencies that include abortion training, there are a number of factors that both enable the training and are barriers to effective training.
Enablers of training, identified in a qualitative survey of participants in OB/GYN residency in New York City (NYC), include: program/facility location, access to high-capacity services, faculty commitment to training, Ryan Program \autocite{bixby_center_for_global_reproductive_health_education_????} \autocite{bixby_center_for_global_reproductive_health_home_????} \autocite{steinauer_benefits_2013} support, and interest of residents. \autocite[pg. 240--2]{guiahi_enablers_2013}
Identified barriers to training include: poor continuity in leadership, conflict between leaders, lacking exposure to second-trimester abortion, difficulties acquiring mifepristone, burden and incompleteness of opt-in training, and antiabortion values of leadership/staff. \autocite[pg. 242]{guiahi_enablers_2013}
While the author emphasizes that these results are not generalizable due to the high-density, urban nature of the survey and the ``politically favorable environment'' experienced in NYC. \autocite[pg. 243]{guiahi_enablers_2013}

The significance of the burden caused by opt-in abortion training noted in the survey by \citeauthor{guiahi_enablers_2013} is identified elsewhere by the practical arrangements required.
Nationally, elective abortion training in the United States was most frequency provided (for 48\% of residents surveyed in 2010) in freestanding clinics outside of the teaching hospital attended by the residents. \autocite[pg. 274]{turk_availability_2014} This indicates the continued separation of both the procedure and training of abortion from the usual domain of hospital-based medicine. Incomplete integration into education is an unfortunate loss for the OB/GYN residents.
While training in therapeutic abortion was available for the vast majority of residents offered either opt-in or no training in elective abortion, surveyed residents offered opt-in elective abortion training had similar experiences in counseling and indicated similar counseling and procedural skills as residents at programs with no elective abortion training. \autocite[pg. 275--6]{turk_availability_2014}
Lacking the positive result is not a reason to curtail abortion training, but an indication that comprehensive training is required to meet the intended results of the ACGME mandate on access to abortion training.

% end *training* portion

% Barriers preventing physicians from providing/offering abortions
\section*{Ability to Provide Abortion Services}

The effect of abortion training during OB/GYN residency is not complete unless the physician can choose to offer and perform abortions, elective or procedural, as they feel is morally and medically necessary.
Despite advocacy for abortion to be integrated into full-spectrum obstetrics and gynecology and primary care, such integration is rare. \autocite[pg. 146]{freedman_obstacles_2010} There are a number of possible professional barriers encountered in integrating even professional discussion of abortion into professional practices, without considering providing the procedure itself.
Two physicians in a 2006 study of physicians who had gone into practice after graduating from programs that offered abortion training encountered threatening anti-abortion sentiments during interviews at professional practices. \autocite[pg. 148]{freedman_obstacles_2010}
Other physicians in the study discovered restrictions on providing abortions after being hired from onerous committee or signatory requirements indicating the ``necessity'' of the abortion to staff or senior partners opposing abortion, requiring referrals to other organizations for the procedure. \autocite[pg. 148]{freedman_obstacles_2010}
These disagreements are more than just ideological differences because they effect the professional paths of both junior and experienced physicians.
Relationships between physicians come to a serious point of conflict on providing or referring for abortions where prochoice physicians make effort to avoid the topic and there is anecdotal evidence that practices have split over the issue. \autocite[pg. 149]{freedman_obstacles_2010}

Beyond peers, larger and more formalized organizational policies prevent some physicians from providing abortions.
The United States is currently in a transitional period of physicians' attitudes towards abortions.
From a survey in 2008--9, younger (26--35 years old) and near-retirement (56--65 years old) OB/GYNs were more likely to provide abortions than middle age (36--45; 46--55 years old) cohorts, with the younger cohort being most likely to provide abortions. \autocite[pg. 611]{stulberg_abortion_2011}
However, these younger physicians will experience barriers to providing abortion not just from their peers, but also from the organizations that employ them.
Catholic health networks can prevent private, independent practices using the network's buildings from providing abortions while nonsectarian health maintenance organizations may mandate abortion referrals for economic reasons rather than providing in-network care. \autocite[pg. 149]{freedman_obstacles_2010}
This religious-based policy might be less harmful to access if Catholic healthcare organizations were less prevalent, but the presence of Catholic healthcare in the United States is significant.
The Catholic Health Association of the United States states that Catholic-affiliated providers are, ``nation’s largest group of not-for-profit health care providers,'' are present in all 50 states, and ``one-in-six patients in the U.S. is cared for in a Catholic hospital.''  \autocite{the_catholic_health_association_of_the_united_states_catholic_2016}
Working primarily in a Catholic medical facility is associated strongly with decreased likelihood of performing abortions while facilities of other religious dominations did not have a significantly different likelihood of performing abortions compared to non-religious facilities. \autocite[pg. 612]{stulberg_abortion_2011}
Both the relationships between physicians and their peers and physicians and the policies of related healthcare organizations prevent physicians from making the individual choices in care they are qualified to choose to offer their patiens.

% Relationships between patients and physicians
\section*{Patient/Physician Relationships and Abortion}

Communication about abortion care is also a critical, though frequently absent or avoided, piece of the patient/physician relationship surrounding female reproductive healthcare. In this relationship, it is the physician's duty to ensure their values are known \textit{and} provide the appropriate medical care for their patient.
It is the opinion of the American College of Obstetricians and Gynecologists that a physician is obliged to provide accurate information and procedures or referrals to their patients for the health of the patients, particularly in reproductive medicine, even when such information or procedures is in conflict with the physicians personal moral or religious beliefs. \autocite[pg. 1]{_limits_2007}
To better enable a physician's understanding of and ability to communicate about their understanding of abortion, values clarification curricula in residency programs provide later benefits for both physician-physician and physician-patient interaction. \autocite[pg. 150]{freedman_obstacles_2010}

For regular healthcare providers, starting the discussion of abortion with patients is valuable well in advance of the patient desiring the procedure.
Due to medical documentation requirements and the separation of abortion from general female healthcare, the disclosure situation can be complicated and concerning for patients. \autocite[pg. 410, 412]{weitz_abortion_2010}
The perception of abortion in the context of regular care is mixed among women seeking an abortion.
In a series of interviews conducted 2006--7, some women enthusiastically desired availability of abortion from their regular healthcare provider for convenience or cost reasons and others indicated a desire for abortion being provided separately for reasons of specialty knowledge or physicians' motivations. \autocite[pg. 411--2]{weitz_abortion_2010}
The women who did not disclose their desire for an abortion to their regular providers did so because they anticipated a degeneration of their medical relationship associated with the stigma of abortion. \autocite[pg. 412]{weitz_abortion_2010}
This fear was, unfortunately, justified from the respondents who broached the topic in a time of need.
Two of the three women had extremely negative responses from disclosing their desire for an abortion to their providers who were strongly opposed to abortion, in part from the unexpected revelation and in part due to the sudden lack of support from their physician. \autocite[pg. 412-3]{weitz_abortion_2010}

While it is not universally beneficial or desirable for patients regular physicians to provide abortions, it is extremely important for physicians to inform their patients of their policy and views on abortion early in the patient/physician relationship to help avoid significant emotional, physical, and emotional costs for their patients. \autocite[pg. 413]{weitz_abortion_2010}
It is essential that healthcare providers do not assume that patients will not, at some point, desire or consider abortion as a choice in their healthcare.
For the benefit of their patients, providers with religious or moral objections should maintain referral process or proximity to providers who do not share those views. \autocite[pg. 1]{_limits_2007}

% Conclusion
\section*{Conclusion}

Abortion will likely remain a controversial topic in the United States and it is the duty of both OB/GYN residency programs and individual physicians to improve patients' access to abortion procedures. While expressing the safety and success of contemporary abortion procedures likely has little influence on ideological criticisms, acknowledging the safety of this common procedure is important.
In a study of 2,550 patients seeking abortion care from 2001 to 2005 at four family medicine clinical practices and one private office/training site, the rate of successful, uncomplicated abortion procedures were 96.5\% for medication and 99.9\% for aspiration. \autocite[pg. 527]{bennett_early_2009}
Maintaining this level of success in abortion care is partially contingent on maintaining the education and training of new and experienced physicians across the United States. Further, the trained and willing physicians must be able to offer safe abortions or, at the very least, referrals to other, accessible organizations that can provide high quality care. To enable this continued advancement, more universal discourse about abortion as a component of healthcare and the role of physicians in providing that vision of complete healthcare is required among medical graduate education. These physicians also need to engage the larger organizations that implement policies that restrict their agency in providing comprehensive healthcare to their patients.

My summary of topics focused on the most widely available literature without focus on particular issues in either abortion training or access to abortion. First, I would like to note the privilege of geographic mobility: residents and physicians characteristically must travel to their educational institutions or employing practice while the needs of patients remain fairly geographically static. The physicians who moved to areas supportive of abortion for their training would have to contend with the greater barriers to practice in areas with fewer or no providers. My research also does not cover the intersectional issue of gender identity and presentation or the effects of and barriers to medical care for transgender or intersex individuals. Finally, I only discussed the training and disclosure elements of abortion. This omits the counseling and support requirements of patients before and after abortion procedures or the need for nonjudgemental support for patients who decide to continue with their pregnancy.

% Bibliographic material
\newpage
%\nocite{*}
\printbibliography

\end{document}

Post-document snippets

\cite[<prenote>][<postnote>]{<key>}
\cite[<postnote>]{<key>}
