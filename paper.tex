\documentclass[letterpaper, 12pt]{article}
\usepackage[utf8]{inputenc}
\usepackage[cmex10]{amsmath}
\usepackage[margin=1in]{geometry}
\usepackage{indentfirst}
\usepackage{graphicx}
\usepackage{color}
%Setup standard date using \today command:
\renewcommand{\today}{\number\day\space\ifcase\month\or
   January\or February\or March\or April\or May\or June\or
   July\or August\or September\or October\or November\or December\fi
   \space\number\year}
%Vector formatting:
\newcommand{\vect}[1]{\boldsymbol{#1}}


\usepackage[pdftex,
            pdfauthor={Thomas Russell Murphy},
            pdftitle={Medical Training for Abortion Procedures in America},
            pdfsubject={The Subject},
            pdfkeywords={abortion, training, medical training, OB/GYN, obstetrician, gynecology, reproduction, fertility, birth control},
            pdfproducer={\pdftexbanner},
            pdfcreator={{\pdftexbanner} with Biber and a Makefile}]{hyperref}

% Setup Chicago 16th edition, footnotes-mode citations
\usepackage[notes,short,backend=biber]{biblatex-chicago}
\bibliography{exported_items_active_updating}

% Use English babel and fancy quotes
\usepackage[american]{babel}
\usepackage{csquotes}

\title{Medical Training for Abortion Procedures and Physician Provision of Abortions in the United States}
\author{Thomas R. Murphy}
\date{\today\\ANTH 360}

\begin{document}
\maketitle

% Original general thesis: "Abortion training is becoming more available and being completed by more physicians, but this trend needs to continue so that future practicing physicians can provide better access to care, particularly with regard to primary care."
% Revised thesis points:
%   - training is not the only barrier
%   - policy influence on care provided due to individual and organizational beliefs

\section*{Introduction}

% Background on abortion in the United States
% Contextualizing these providers of abortion

Just over a million abortions reported in 2011 \autocite[pg. 7]{jones_abortion_2014}

In 2011, 89\% of US counties had no abortion or nonspecialized clinic with 38\% of women aged 15–44 lived in those counties, essentially unchanged from the geographic availability in 2008. \autocite[pg. 7]{jones_abortion_2014}

While the presence of abortion providers is more concentrated in cities, in 2008, 69\% of counties in metropolitan areas lacked a provider while 97\% of nonmetropolitan counties lacked a provider. \autocite[pg 46]{jones_abortion_2011}

A 2008-9 survey found that 97\% of obstetrician-gynecologists in the United States under the age of 65 encountered patients seeking abortions while only 14.4\% of those practicing ob-gyns provided abortions themselves. \autocite[pg. 611]{stulberg_abortion_2011}

While 19\% of abortion providers in 2008 were physicians' offices, they only provided 1\% of all abortions at a low rate per location and were the most expensive. \autocite[pg. 46-7]{jones_abortion_2011}

32.8\% of births were Cesarean deliveries, about 1.3 million births in 2011 \autocite[pg. 5]{martin_births_2013}

20,090 obstetricians and gynecologists employed in May 2015, primarily physicians' offices and outpatient care centers \autocite{occupational_employment_and_wages_may_2015}

% Changing environment of abortion healthcare
The decision of \textit{Roe v Wade} enabled abortions to shift out of hospitals and into non-hospital clinics, thus removing the common procedure and its associated curricula from the US medical educational environment. \autocite[pg. 404]{aksel_unintended_2013}

% Discussion of OB/GYN *training* in residency
\section*{Abortion Training Access in OB/GYN Residency}

The early 1990s saw a marked lack of training in family planning, including abortion, but in 1994 Jody Steinauer started the meetings with fellow medical students that became the organization Medical Students for Choice. \autocite[pg. 404-5]{aksel_unintended_2013}

% Enablers, barriers, consequences
training is required \autocite[IV.A.6.d, pg. 17]{accreditation_council_for_graduate_medical_education_acgme_2014}
First mandated in 1996 \autocite[pg. 146]{freedman_obstacles_2010}

Enablers of training: location, access to high-capacity services, faculty commitment to training, Ryan Program support, and interest of residents. \autocite[pg. 240-2]{guiahi_enablers_2013}

Barriers to training: poor continuity in leadership, conflict in leadership, lacking exposure to second-trimester abortion, difficulties acquiring mifepristone, optional training, antiabortion values of leadership/staff. \autocite[pg. 242]{guiahi_enablers_2013}
% Guiahi: not generalizable, but valuable qualitative starting point

Elective abortion was most frequency provided (for 48\% of residents surveyed in 2010) in freestanding clinics outside of the teaching hospital attended by the residents. \autocite[pg. 274]{turk_availability_2014}

While training in therapeutic abortion was available for the vast majority of residents offered ethier opt-in or no training in elective abortion, surveyed residents offered opt-in elective abortion training had similar experiences in counseling and indicated similar counseling and procedural skills as residents at programs with no elective abortion training. \autocite[pg 275-6]{turk_availability_2014}

% end *training* portion

% Barriers preventing physicians from providing/offering abortions
\section*{Ability to Provide Abortion Services}

Despite advocacy for abortion to be integrated into full-spectrum obstetrics and gynecology and primary care, such integration is rare. \autocite[pg 146]{freedman_obstacles_2010}

Two physicians in a 2006 study of physicians who had gone into practice after graduating from programs that offered abortion training encountered threatening anti-abortion sentiments during interviews at professional practices. \autocite[pg 148]{freedman_obstacles_2010}

Other physicians in the study discovered restrictions on providing abortions after being hired from onerous committee or signatory requirements indicating the ``necessity'' of the abortion to staff or senior partners opposing abortion, requiring referrals to other organizations for the procedure. \autocite[pg 148]{freedman_obstacles_2010}

Relationships between physicians come to a fairly serious point of conflict on providing or referring for abortions where prochoice physicians make effort to avoid the topic and anecdotal evidence that practices have split over the issue. \autocite[pg 149]{freedman_obstacles_2010}

Beyond peers, larger and more formalized organizational policies prevent some physicians from providing abortions. Catholic health networks can prevent private, independent practices using the network's buildings from providing abortions while nonsectarian health maintenance organizations may mandate abortion referrals for economic reasons rather than providing in-network care. \autocite[pg 149]{freedman_obstacles_2010}

Younger (26-35 years old) and near-retirement (56-65 years old) OB/GYNs are more likely to provide abortions than middle age (36-45; 46-55 years old) cohorts, with the younger cohort being most likely to provide abortions. \autocite[pg. 611]{stulberg_abortion_2011}

Working primarily in a Catholic medical facility is associated strongly with decreased likelihood of performing abortions while facilities of other religious dominations did not have a significantly different likelihood of performing abortions compared to non-religious facilities. \autocite[pg. 612]{stulberg_abortion_2011}

% Relationships between patients and physicians
\section*{Patient/Physician Relationships and Abortion}


mandating of limits of conscientious refusal \autocite[]{_limits_2007}

Values clarification curricula in residency programs provide benefits for both physician-physician and physician-patient interaction. \autocite[pg 150]{freedman_obstacles_2010}

Due to medical documentation requirements and the separation of abortion from general female healthcare, the disclosure situation can be complicated and concerning for patients. \autocite[pg. 410, 412]{weitz_abortion_2010}

In a series of interviews conducted 2006-7, some women enthusiastically desired availability of abortion from their regular healthcare provider for convenience or cost reasons and others indicated a desire for abortion being provided separately for reasons of specialty knowledge or physicians' motivations. \autocite[pg. 411-2]{weitz_abortion_2010}

Women who did not disclose their desire for an abortion to their regular providers did so because they anticipated a degeneration of their medical relationship associated with the stigma of abortion. \autocite[pg. 412]{weitz_abortion_2010}

Two of the three women had extremely negative responses from disclosing their desire for an abortion to their providers who were strongly opposed to abortion, in part from the unexpected revelation and in part due to the sudden lack of support from their physician. \autocite[pg. 412-3]{weitz_abortion_2010}

While it is not universally beneficial or desirable for patients regular physicians to provide abortions, it is extremely important for physicians to inform their patients of their policy and views on abortion early in the patient/physician relationship to help avoid significant emotional, physical, and emotional costs for their patients. \autocite[pg. 413]{weitz_abortion_2010}

Don't assume that patients will not, at some point, desire or consider abortion as a choice in their healthcare

% Conclusion
\section{Conclusion}

In a study of 2,550 patients seeking abortion care from 2001 to 2005 at four family medicine clinical practices and one private office/training site, the rate of successful, uncomplicated abortion procedures were 96.5\% for medication and 99.9\% for aspiration. \autocite[pg. 527]{bennett_early_2009}


% note privilege of geographic mobility
% Does not cover intersectional issue of gender identity and presentation or effects of medical care for transgender or intersex individuals

% Bibliographic material
\newpage
%\nocite{*}
\printbibliography

\end{document}

Post-document snippets

\cite[<prenote>][<postnote>]{<key>}
\cite[<postnote>]{<key>}
