\documentclass[letterpaper, 12pt]{article}
\usepackage[utf8]{inputenc}
\usepackage{amsmath}
\usepackage{fullpage}
\usepackage{mla} %Requires having some version of mla.sty in the same directory
%Setup standard date using \today command:
\renewcommand{\today}{\number\day\space\ifcase\month\or
   January\or February\or March\or April\or May\or June\or
   July\or August\or September\or October\or November\or December\fi
   \space\number\year}
\setlength{\headheight}{15pt}

\usepackage[american]{babel}
\usepackage{csquotes}
\usepackage[style=mla,backend=biber]{biblatex}
\bibliography{exported_items.bib}

\DefineBibliographyStrings{english}{%
 bibliography = {},%
 references= {},%
}

\begin{document}

\begin{mla}{Thomas}{Murphy}{Dr. Lihong Shi}{ANTH 360}{15 March 2016}{Research Paper Proposal: Medical Training in Abortion Procedures}

One of the fundamental factors determining the accessibility of abortion is the presence of qualified, trained physicians to perform abortion-related procedures. From the 1980s to present, there has been a marked decline in the number of abortion providers in the United States. One frequently cited statistic is that 87\% of all US counties do not have an abortion provider, leaving around a third of often-rural women needing to travel for such a procedure. In addition, during the 1990s, there was a distinct shortage of physicians able and willing to perform abortions in the United States as a consequence of the aging population of abortion-performing physicians and the lack of training for new physicians in relevant areas of medicine.

With this context, I will be looking to establish the general thesis that abortion training is becoming more available and being completed by more physicians, but this trend needs to continue so that future practicing physicians can provide better access to care, particularly with regard to primary care. To support this thesis, I will establish the historical trends in abortion training, as briefly described above, and then discuss current training, the motivations of residents and institutions involved in training, and discuss the consequences of participation in abortion care during residency. To reach the conclusion of this thesis, I will discuss the advantages of more abortion providers and the availability of abortion through primary care providers to people with the ability to become pregnant. Finally, to contrast this perspective with the current environment of care, I will mention the systemic problems of funding and state restriction, both for individual operations and the availability of comprehensive training.


\begin{workscited}

\nocite{*}
\vspace*{-12ex}
\printbibliography[title={}]

\end{workscited}

\end{mla}
\end{document}
